\documentclass{article}
\usepackage[UTF8]{ctex}
\usepackage{amsfonts}
\usepackage{amsmath}
\usepackage{bm}

\newtheorem{problem}{问题}

\title{题目待定}
\author{张瑜安\\ 计算机科学与技术学院 \\21921060}
\date{}
\begin{document}
\maketitle
\section{介绍}
内点法是求解凸优化问题非常优秀的一类方法,其做法是通过在约束可行域中的一点出发,
迭代探索附近的点来逼近最优解,这也是“内点”这个名字的含义。在线性规划问题的求解上,
内点法比单纯形法要快,而且内点法一般保证了多项式复杂度,单纯形法则没有。
内点法也被拓展到更加一般的凸优化问题当中(ref 两个姓N的人的研究)。
到目前为止,内点法是相当流行的凸优化问题求解方法,主流的求解器都会集成这类方法。实际
使用的时候,它的速度很快,迭代次数也很少。但它也有一定的局限性,一是对函数进行了
二次可微的限制(ref 找一找权威的),相比于使用次梯度的方法适用范围更窄一些,
二是它是一个点方法,没法求解一个极值的区域(是不是改成局部方法无法取到最小值好一点?)。

内点法的一般思路是使用可微的障碍函数(Barrier Function)代替一个理想的惩罚函数,
同时将优化问题变成求解最优条件的问题,就可以使用牛顿法来解。本文介绍两种思路相承接的方法,
第一类是直接应用障碍函数、从中心路径逼近的方法,第二类是考虑对偶变量一起优化的原始
对偶方法。其实后者是前者的一个扩展,在ref(方法)一节中将会展示如何从第一种方法
启发得到第二种,这种启示思路来自于ref(Convex Optimization)一书。


\section{背景调研}
(这一段参考找The interior-point revolution in optimization: History, recent developments, and lasting consequences里的)
内点法起源可以追溯到1955年,(ref the logarithm potential method of convex 
programming)一书就在用障碍函数做凸优化。到上世纪六十年代,内点法主要被应用于非线
性的优化,因为当时单纯形法在线性规划中拥有不可撼动的地位。到七十年代内点法的研究
越来越少,其中一个原因是这些方法依赖的关键矩阵是病态的,方法的稳定性弱。

到八十年代初内点法差不多就要盖棺定论了,然而1984年Karmarkar在线性规划领域提出了
一个多项式时间复杂度的算法,实际效率比单纯形法快五十倍,后续研究表明该方法和以前用
障碍函数的方法本质上十分接近,从而内点法重新成为研究热点。

朴素的利用障碍函数的方法
仍然存在效率问题,最重要的原因是一个固定的障碍参数(类似于拉格朗日乘子,在方法
(ref)中提到)对应了一轮牛顿迭代,障碍参数的外层迭代没法和内层迭代适应性地变化。
所以后续又有许多相似算法被提出来,其中最受瞩目的是原始对偶方法,它使固定的“障碍参数”
变成可迭代的对偶参数,从而将原先内外两层迭代融合成一层,效率更上一层楼。
到今天为止原始对偶内点法仍然具有相当的竞争力,在选用内点法时都会优先考虑这类方法。


\section{报告组织}
本文第一章对内点法做了简单介绍,第二章介绍了内点法的历史发展,这些内容主要来自于Wikipedia和ref(IPM:A survey, short survey The interior-point revolution in optimization: History, recent developments, and lasting consequences...)等几篇综述。
第三章就是本章介绍文章结构,第四章约定记号并简述本文关注的凸优化问题及对偶问题、KKT条件和牛顿法等必要知识。

第五章首先介绍了重要工具障碍函数,然后介绍了直接使用障碍函数的朴素方法。然后介绍了
原始对偶方法,并且说明了原始对偶方法可以如何从障碍函数启发得到。第六章对第五章中的方法进行了可行性和算法复杂度分析,并结合实验(?待定)分析其优劣。
这两章内容主要来自于ref(Convex Optimization)和ref2(Convex Optimization:Algorithms and Complexity)两本书。

第七章介绍了内点法的应用,最后,第九章对内点法做了总结。
\section{记号和预备知识}
\subsection{记号}
$\mathbb{R}$ $\mathbb{R}_+$
$\bm x$ $\textbf{dom}f$
$\succeq$
$n$维数 $m$约束个数
\subsection{预备知识}
本文考虑的凸优化问题具有如下形式:
\begin{problem}
最小化$f_0(\bm x)$,使得$f_i(\bm x)\le 0(i=1,2,...,m),A\bm x=\bm{b}$成立,其中$f_i(\bm x):\mathbb{R}^n\rightarrow\mathbb{R}(i=0,1,...,m)$均为凸函数,$A\in\mathbb{R}^{p\times n},\bm b\in\mathbb{R}^p$。
\label{general_convex_prob}
\end{problem}

上述问题中$\bm x$的维数是$n$,约束一共是$m+p$个,其中有$p$个是等式线性约束。
记定义域$\mathcal{D}=\bigcap\limits_{i=0}^{m}\textbf{dom}f_i$,
定义该问题的拉格朗日对偶函数为:
\begin{equation}
    g(\bm\lambda, \bm\nu)=\inf_{\bm x\in D}(f_0(\bm x)+\sum\limits_{i=1}^m\lambda_if_i(\bm x)+\bm\nu^T(A\bm x-\bm b))\label{dual_function}
\end{equation}
其中$\bm\lambda=(\lambda_1,\lambda_2,...,\lambda_m)\in\mathbb R^m,\bm\nu\in\mathbb{R}^p$,那么问题\ref{general_convex_prob}的对偶问题为:
\begin{problem}
最大化$g(\bm\lambda,\bm\nu)$,使得$\bm\lambda\succeq\bm 0$。
\end{problem}
\section{方法}
\subsection{障碍函数}
\subsection{障碍方法}
\subsection{原始对偶方法}
\section{理论分析}
\section{应用}
\section{总结}
\end{document}