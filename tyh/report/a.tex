\documentclass{article}
\usepackage[UTF8]{ctex}
\usepackage{amsfonts}
\usepackage{amsmath}
\usepackage[ruled,vlined]{algorithm2e}
\usepackage{bm}

\newtheorem{problem}{问题}

\title{题目待定}
\author{张瑜安\\ 计算机科学与技术学院 \\21921060}
\date{}
\begin{document}
\maketitle
\section{介绍}
内点法是求解凸优化问题非常优秀的一类方法,其做法是通过在约束可行域中的一点出发,
迭代探索附近的点来逼近最优解,这也是“内点”这个名字的含义。在线性规划问题的求解上,
内点法比单纯形法要快,而且内点法一般保证了多项式复杂度,单纯形法则没有。
内点法也被拓展到更加一般的凸优化问题当中(ref 两个姓N的人的研究)。
到目前为止,内点法是相当流行的凸优化问题求解方法,主流的求解器都会集成这类方法。实际
使用的时候,它的速度很快,迭代次数也很少。但它也有一定的局限性,一是对函数进行了
二次可微的限制(ref 找一找权威的),相比于使用次梯度的方法适用范围更窄一些,
二是它是一个点方法,没法求解一个极值的区域(是不是改成局部方法无法取到最小值好一点?)。

内点法的一般思路是使用可微的障碍函数(Barrier Function)代替一个理想的惩罚函数,
同时将优化问题变成求解最优条件的问题,就可以使用牛顿法来解。本文介绍两种思路相承接的方法,
第一类是直接应用障碍函数、从中心路径逼近的方法,第二类是考虑对偶变量一起优化的原始
对偶方法。其实后者是前者的一个扩展,在ref(方法)一节中将会展示如何从第一种方法
启发得到第二种,这种启示思路来自于ref(Convex Optimization)一书。


\section{背景调研}
(这一段参考找The interior-point revolution in optimization: History, recent developments, and lasting consequences里的)
内点法起源可以追溯到1955年,(ref the logarithm potential method of convex 
programming)一书就在用障碍函数做凸优化。到上世纪六十年代,内点法主要被应用于非线
性的优化,因为当时单纯形法在线性规划中拥有不可撼动的地位。到七十年代内点法的研究
越来越少,其原因未知,猜测其中之一是接近障碍函数边缘时,
这些方法依赖的关键矩阵是病态的,方法的稳定性弱。

到八十年代初内点法差不多就要盖棺定论了,然而1984年Karmarkar在线性规划领域提出了
一个多项式时间复杂度的算法,实际效率比单纯形法快五十倍,后续研究表明该方法和以前用
障碍函数的方法本质上十分接近,从而内点法重新成为研究热点。

朴素的利用障碍函数的方法存在效率问题,最重要的原因是一个
固定的障碍参数对应了一轮牛顿迭代,障碍参数的外层迭代没法和内层迭代适应性地变化。
所以后续又有许多相似算法被提出来,其中最受瞩目的是原始对偶方法,它使固定的“障碍参数”
变成可迭代的对偶参数,从而将原先内外两层迭代融合成一层,效率更上一层楼。
到今天为止原始对偶内点法仍然具有相当的竞争力,在选用内点法时都会优先考虑这类方法。


\section{报告组织}
本文第一章对内点法做了简单介绍,第二章介绍了内点法的历史发展,这些内容主要来自于Wikipedia和ref(IPM:A survey, short survey The interior-point revolution in optimization: History, recent developments, and lasting consequences...)等几篇综述。
第三章就是本章介绍文章结构,第四章约定记号并简述本文关注的凸优化问题、对偶问题和KKT条件等必要知识。

第五章首先介绍了重要工具障碍函数,然后介绍了直接使用障碍函数的朴素方法。然后介绍了
原始对偶方法,并且说明了原始对偶方法可以如何从障碍函数启发得到。第六章对第五章中的方法进行了可行性和算法复杂度分析,并结合实验(?待定)分析其优劣。
这两章内容主要来自于ref(Convex Optimization)和ref2(Convex Optimization:Algorithms and Complexity)两本书。

第七章介绍了内点法的应用,最后,第九章对内点法做了总结。
\section{记号和预备知识}
\subsection{记号}
$\mathbb{R}$ $\mathbb{R}_+$
$\bm x$ $\textbf{dom}f$
$\succeq$
$\nabla$
$n$维数 $m$约束个数
$\textbf{f}(\bm x)$函数$\mathbb{R}^n\rightarrow\mathbb{R}^m$
$\text{D}\textbf{f}(\bm x)$雅可比矩阵1
$\textbf{diag}$
\subsection{预备知识}
本文考虑的凸优化问题具有如下形式:
\begin{problem}
最小化$f_0(\bm x)$,使得$f_i(\bm x)\le 0(i=1,2,...,m),A\bm x=\bm{b}$成立,其中$f_i(\bm x):\mathbb{R}^n\rightarrow\mathbb{R}(i=0,1,...,m)$均为凸函数,$A\in\mathbb{R}^{p\times n},\bm b\in\mathbb{R}^p$。
\label{general_convex_prob}
\end{problem}

上述问题中$\bm x$的维数是$n$,约束一共是$m+p$个,其中有$p$个是等式线性约束。
记定义域$\mathcal{D}=\bigcap\limits_{i=0}^{m}\textbf{dom}f_i$,
定义该问题的拉格朗日函数和对偶函数为:
\begin{equation}
    L(\bm x,\bm\lambda, \bm\nu)=f_0(\bm x)+\sum\limits_{i=1}^m\lambda_if_i(\bm x)+\bm\nu^T(A\bm x-\bm b)\label{lagrange_function}
\end{equation}
\begin{equation}
    g(\bm\lambda, \bm\nu)=\inf_{\bm x\in D}\{L(\bm x,\bm\lambda, \bm\nu)\}\label{dual_function}
\end{equation}
其中$\bm\lambda=(\lambda_1,\lambda_2,...,\lambda_m)\in\mathbb R^m,\bm\nu\in\mathbb{R}^p$,那么问题\ref{general_convex_prob}的对偶问题为:
\begin{problem}
最大化$g(\bm\lambda,\bm\nu)$,使得$\bm\lambda\succeq\bm 0$。\label{dual_problem}
\end{problem}

记问题\ref{general_convex_prob}的最优值为$p^*$,问题\ref{dual_problem}的最优值为$d^*$,
那么$d^*\le p^*$。使$d^*=p^*$成立的一个条件是\textbf{Slater条件}:存在一点$\bar{\bm x}\in \textbf{relint}\mathcal D$使得
$$f_i(\bar{\bm x})<0(i=1,...,m),A\bar{\bm x}=\bm b$$
成立。现在假设$f_i(\bm x)(i=0,1,...,m)$\textbf{可微},那么可以推导出\textbf{Karush-Kuhn-Tucker条件},即当存在$\bm x^*,\bm\lambda^*,\bm\nu^*$满足
\begin{equation}
    \label{kkt}
    \begin{gathered}
    A\bm x^*=\bm b,f_i(\bm x^*)\le 0,i=1,...,m \\
    \bm\lambda^*\succeq \bm 0 \\
    \lambda^*_if_i(\bm x^*)=0,i=1,...,m\\
    \nabla f_0(\bm x^*)+\sum_{i=1}^m{\lambda^*_i}\nabla f_i(\bm x^*)+A^T\bm\nu^*=\bm 0\\
    \end{gathered}
\end{equation}
时,$d^*=p^*$,并且$\bm x^*,(\bm\lambda^*,\bm\mu^*)$分别是原问题和对偶问题的最优解。

不考虑不等式约束,仅考虑等式约束,那么上述KKT条件简化为:
\begin{equation}
    A\bm x^*=\bm b,\nabla f_0(x^*)+A^T\bm\nu ^*=\bm 0\label{equation_constraint_kkt}
\end{equation}

加强$f_0(\bm x)$,假设其\textbf{二阶可微},那么就可以用牛顿法求解约束\eqref{equation_constraint_kkt}中的
$\bm x^*$。用$\bm x+\Delta \bm x$代替式\eqref{equation_constraint_kkt}
中的$\bm x^*$,并对$\nabla f_0(\bm x+\Delta \bm x)$做一阶近似就可以得到
计算迭代方向$\Delta \bm x$的方程:
\begin{equation}
\begin{pmatrix}
    \nabla^2f_0(\bm x) & A^T\\
    A & \bm 0
\end{pmatrix}
\begin{pmatrix}
    \Delta \bm x\\
    \bm\nu^*
\end{pmatrix}=
\begin{pmatrix}
    -\Delta f_0(\bm x)\\
    \bm 0
\end{pmatrix}\label{newton_for_equation}\end{equation}
如果我们能找到一个初始可行点$\bm x_0$,就可以用$x_k=x_{k-1}+\mu \Delta \bm x$来求解$\bm x^*$。到此为止,
我们有了一个求解仅含等式约束的凸优化问题的牛顿方法。
下一章将会介绍如何利用障碍函数将不等式约束融合进目标函数,从而可以使用此处所描述
的牛顿法求解。
\label{text4_2}
\section{方法}
\subsection{障碍函数和中心路径}
我们试图将问题\ref{general_convex_prob}转换为等式约束问题,因为在\ref{text4_2}
中已经提出了一个解决等式约束问题的牛顿方法。引入如下函数:
$$I_-(u)=\begin{cases}
    0 & u\le 0 \\
    \infty & u > 0
\end{cases}$$
用该函数来惩罚大于零的$f_i(\bm x),1\le i\le m$,那么问题\ref{general_convex_prob}等价于:
\begin{problem}
最小化$f_0(\bm x)+\sum_{i=1}^m{I_-(f_i(\bm x))}$使得$A\bm x=\bm b$
\label{barrier_problem}
\end{problem}

问题\ref{barrier_problem}中,需要优化的目标函数在不等式约束内等于$f_0(\bm x)$,
其他区域内没有定义,所以和问题\ref{general_convex_prob}的等价是显然的。

引入$I_-(u)$后,不等式约束被融合进目标函数,只剩下等式约束。
然而问题\ref{barrier_problem}中目标函数是一个不可微的函数,我们需要做一点近似才能应用\ref{text4_2}中
的牛顿方法。用
$$\widehat I_-(u)=-(1/t)\log(-u)$$
近似$I_-(u)$(其中$t>0$为常数),问题\ref{barrier_problem}可以近似成如下问题:
\begin{problem}
    最小化$f_0(\bm x)+\sum_{i=1}^m{-(1/t)\log(f_i(\bm x))}$使得$A\bm x=\bm b$
    \label{barrier_approx_prob}
\end{problem}

可以看出,$t$越大,$\widehat I_-(u)$越接近$I_-(u)$,问题\ref{barrier_approx_prob}
和问题\ref{barrier_problem}的目标函数就越接近,其解也越接近。
目标函数中的对数惩罚部分$\phi(\bm x)=-\sum_{i=1}^m\log (-f_i(\bm x))$被称做问题\ref{general_convex_prob}的\textbf{对数障碍函数}。

当$f_i(\bm x),i=0,...,m$\textbf{二阶可微}时,问题\ref{barrier_approx_prob}就可以应用\ref{text4_2}中
的牛顿方法。一个朴素的思路是直接设一个很大$t$,计算在$t$下问题$\ref{barrier_approx_prob}$的解。
但是$t$很大时,目标函数的Hessian矩阵在可行域边界剧烈变动。所以在\ref{text_barrier_method}中
的方法考虑逐渐增加$t$,解决一系列的问题\ref{barrier_approx_prob}来规避这个问题。

简化符号,重写问题\ref{barrier_approx_prob}:
\begin{problem}
    最小化$f_0(\bm x)+(1/t)\phi(\bm x)$,使得$A\bm x=\bm b$
    \label{barrier_approx_prob_simple}
\end{problem}

对某一$t>0$,用\textbf{中心点}$\bm x^*(t)$表示问题\ref{barrier_approx_prob_simple}的解,
将不同$t$对应的中心点集合称为问题\ref{general_convex_prob}的\textbf{中心路径}。
中心路径上的点需在障碍函数的定义域中,并且满足
式\eqref{equation_constraint_kkt}的KKT条件,即:
$$A\bm x^*(t)=\bm b,f_i(\bm x^*(t))<0,i=1,...,m$$
并且存在$\widehat{\bm \nu}\in \mathbb{R}^p$使得
\begin{equation}\nabla f_0(\bm x^*(t))+(1/t)\nabla\phi(\bm x^*(t))+A^T\widehat{\bm \nu}=\bm 0\label{center_path_kkt}\end{equation}
把式\eqref{center_path_kkt}中$\phi$的梯度展开写就是:
\begin{equation}
    \nabla f_0(\bm x^*(t))+\sum_{i=1}^m{\frac{1}{-tf_i(\bm x^*(t))}\nabla f_i(\bm x^*(t)}+A^T\widehat{\bm \nu}=\bm 0\label{center_path_kkt_2}
\end{equation}

在\ref{time_complexity}将会证明$f_0(\bm x^*(t))-p^*\le m/t$,即中心点的函数值
和最优解相差不超过$m/t$,从而我们可以在求解过程中对精度进行控制。
从该结论也可以得出当$t\rightarrow\infty$时,$\bm x^*(t)$从中心路径逼近原问题的解。下一节介绍从中心路径逼近最优解的方法。
\label{text_barrier_center_path}
\subsection{障碍函数法}
\label{text_barrier_method}
有了\ref{text_barrier_center_path}的铺垫,很自然的可以得到如下算法:

\renewcommand{\algorithmcfname}{算法}
\begin{algorithm}[H]
    % \SetKwInOut{KIN}{输入}
    % \SetKwInOut{KOUT}{输出}
    \KwIn {严格可行点$\bm x_0$,参数$t>0,\mu>1$,误差阈值$\epsilon>0$}
    \For {$i \gets 0\dots \infty $} {
        从$\bm x_i$出发,在$A\bm x=\bm b$约束下极小化$f_0(\bm x)+(1/t)\phi(\bm x)$,解出中心点$\bm x_i^*(t)$\;
        $\bm x_{i+1}\leftarrow\bm x_i^*(t)$\;
        \If{$m/t_i<\epsilon$} {
            \Return $\bm x_{i+1}$
        }
        $t\leftarrow \mu t$
    }
    \caption{障碍函数法}
    \label{barrier_method}
\end{algorithm}

此处对算法\ref{barrier_method}中的重要步骤进行两点说明:
\begin{enumerate}
    \item 解中心点$\bm x_i^*(t)$是仅有线性等式约束的凸优化问题,解这类问题的方法
    已经在$\ref{text4_2}$描述过了,使用的是牛顿法。因而该算法存在内外两层迭代。
    \item 初始的严格可行点$\bm x_0$可以通过求解问题\ref{prepare}找到,此问题和我们要求解
    的问题是同一类问题,不同之处在于该问题的初始可行解很容易找,
    因为任取定义域中满足$A\bm x=\bm b$的一点$\bm x_0$,总能找到足够大的$s_0$与$x_0$一起
    作为初始解。求解严格可行点$x_0$还有其他方法,比如使用不可行初始点牛顿法,
    需要修改式\ref{newton_for_equation},此处不再赘述。
    \begin{problem}
        最小化$s$,使得$f_i(\bm x)\le s,i=1,...,m,A\bm x=\bm b$
        \label{prepare}
    \end{problem}
\end{enumerate} 
对算法的参数$t$和$\mu$进行简单分析:
\begin{enumerate}
    \item 如果$\mu$较小,$t$变化之后,$x^*(t)$变化也较小,所以上一次迭代终点会是这次
    迭代很好的初始点。这样内层的牛顿迭代次数会很少,但是外层的$t$达到$m/\epsilon$要经过很多次的迭代。
    反之,如果$\mu$较大,内层迭代次数偏多,外层迭代次数减少。
    不过$\mu$对总的迭代次数影响并不是很大,实践中可以调参。
    \item 外层参数$t$的迭代无法和内层迭代一起适应性的变化,即$t$必须等到一次牛顿迭代完成后才能变化。这是该算法的效率瓶颈之一。
\end{enumerate}
\subsection{原始对偶方法}
\label{prim_dual_method}
中心路径和对偶问题有着十分紧密的联系。对固定的$t$,中心点是$\bm x^*(t)$,
取$\bm\lambda_i^*(t)=-\frac{1}{tf_i(\bm x^*(t))},i=1,...,m$,令$\bm \nu^*(t)$为式\eqref{center_path_kkt}中的$\widehat{\bm\nu}$。
根据式\eqref{center_path_kkt_2}有:
$$\nabla f_0(\bm x^*(t))+\sum_{i=1}^m{\bm\lambda_i^*(t)\nabla f_i(\bm x^*(t)}+A^T\bm \nu^*(t)=\bm 0$$
考虑问题\ref{general_convex_prob}的拉格朗日函数式\eqref{lagrange_function},其关于$\bm x$的导数为:
$$\frac{\partial L(\bm x,\bm \lambda,\bm \nu)}{\partial\bm x}=
\nabla f_0(\bm x)+\sum_{i=1}^m{\bm\lambda_i\nabla f_i(\bm x)}+A^T\bm\nu$$
代入$\bm x^*(t),\bm\lambda_i^*(t),\bm\nu^*(t)$恰好为零,说明拉格朗日函数在
$\bm x^*(t)$取极大值,这样就可以求得
\begin{flalign}
    \begin{aligned}
    g(\bm\lambda^*(t),\bm\nu^*(t))&=f_0(\bm x^*(t))+\sum_{i=1}^m{\bm\lambda_i^*(t)f_i(\bm x^*(t))}+\bm\nu^*(t)^T(A\bm x^*(t)-\bm b)\\
    &=f_0(\bm x^*(t))-m/t
    \end{aligned}&&&
\end{flalign}

由$g(\bm\lambda^*(t),\bm\nu^*(t))\le d*=p*\le f_0(\bm x^*(t))$,
可知当$t\rightarrow\infty$,$\bm x^*(t)$从中心路径收敛于最优解的同时,$\bm\lambda^*(t),\bm\nu^*(t)$也收敛于
对偶问题的最优解,$f_0(\bm x^*(t))$与$g(\bm\lambda^*(t),\bm\nu^*(t))$分别作为上下界逼近最优解。由此得到启发,可以同时迭代$\bm x,\bm\lambda,\bm\nu$来接近
最优解。

对KKT条件式\eqref{kkt}进行修改得到式\eqref{kkt_modified}。式\eqref{kkt_modified}与\eqref{kkt}的唯一区别
就是第三组等式$\lambda_if_i(\bm x)=0$改成了$\lambda_if_i(\bm x)=1/t$。
这是从中心点和其对偶点的关系启发得到的。
\begin{equation}
    \label{kkt_modified}
    \begin{gathered}
    A\bm x=\bm b,f_i(\bm x)\le 0,i=1,...,m \\
    \bm\lambda\succeq \bm 0 \\
    \lambda_if_i(\bm x)=1/t,i=1,...,m\\
    \nabla f_0(\bm x)+\sum_{i=1}^m{\lambda_i}\nabla f_i(\bm x)+A^T\bm\nu=\bm 0\\
    \end{gathered}
\end{equation}
现在希望$\bm\lambda,\bm\nu$能和$\bm x$一起用牛顿法迭代求解。为方便表述,
将式\eqref{kkt_modified}中的等式方程写成$\bm r_t(\bm x,\bm\lambda,\bm\nu)=\bm 0$。$r_t(\bm x,\bm\lambda,\bm\nu)$由
式\eqref{kkt_vector_form}给出:
\begin{equation}
\label{kkt_vector_form}
\bm r_t(\bm x,\bm\lambda,\bm\nu)=\begin{pmatrix}
    \nabla f_0(\bm x)+\text{D}\textbf{f}(\bm x)^T\bm\lambda+A^T\bm\nu \\
    -\textbf{diag}(\bm\lambda)\textbf{f}(\bm x)-(1/t)\bm 1 \\
    A\bm x-\bm b
\end{pmatrix}
\end{equation}
其中$\textbf{f}(\bm x)=\begin{pmatrix}
    f_1(\bm x) \\
    \vdots \\
    f_m(\bm x)
\end{pmatrix}$,$\text{D}\textbf{f}(\bm x)=\begin{pmatrix}
    \nabla f_1(\bm x)^T \\
    \vdots \\
    \nabla f_m(\bm x)^T
\end{pmatrix}$是$\textbf{f}(\bm x)$的雅可比矩阵。

从$(\bm x,\bm\lambda,\bm\nu)$要求迭代方向$(\Delta\bm x,\Delta\bm\lambda,\Delta\bm\nu)$。令$$\bm r_t(\bm x+\Delta\bm x,\bm\lambda+\Delta\bm\lambda,\bm\nu+\Delta\bm\nu)=\bm 0$$
类似于牛顿法的分析,利用$f(\bm x+\Delta\bm x)\approx f(\bm x)+\nabla f(\bm x)\Delta\bm x,\nabla f(\bm x+\Delta \bm x)\approx\nabla f(\bm x)+\nabla^2f(\bm x)\Delta \bm x$的近似,并舍弃带有$\Delta\bm\lambda\Delta\bm x$的项,得到式\eqref{prim_dual_direction}
\begin{equation}
    \label{prim_dual_direction}\begin{pmatrix}
    \nabla^2f_0(\bm x)+\sum\limits_{i=1}^m{\lambda_i\nabla^2f_i(\bm x)} & \text{D}\textbf{f}(\bm x)^T & A^T \\
    -\textbf{diag}(\bm\lambda)\text{D}\textbf{f}(\bm x) & -\textbf{diag}(\textbf{f}(\bm x)) & 0 \\
    A & 0 & 0
\end{pmatrix}\begin{pmatrix}
    \Delta\bm x\\
    \Delta\bm\lambda\\
    \Delta\bm\nu
\end{pmatrix}=-\bm r_t\end{equation}
\section{理论分析}
\subsection{可行性分析}
\subsection{时间复杂度分析}
\label{time_complexity}
$f_0(\bm x^*(t))-p^*\le m/t$
\section{应用}
\section{总结}
\end{document}